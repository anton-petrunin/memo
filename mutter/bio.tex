\section*{Родословная (всё, что знаю)}
\addcontentsline{toc}{section}{Родословная}

Я,  Нина Анатольевна Хоткевич, родилась в Ленинграде 26 июня 1938 года и до начала войны, до 1941 года,  жила  в посёлке Дибуны Ленинградской области в собственном доме, принадлежавшем родителям моей матери.
Адрес дома: улица Железнодорожная, дом 9.
Во взрослом возрасте крестилась в православном соборе иконы Владимирской божьей матери, позднее 1,5 года изучала религию, прошла конфирмацию, осознанно выбрав для себя лютеранскую веру.

Моя мать, Бочарова-Трейман Людмила Александровна  родилась в посёлке Дибуны 28 января 1906 года.
Крестили её в православной церкви Кирилла и Мефодия.
В 1935 году вышла замуж за Бочарова Анатолия Васильевича, но сохранила и свою девичью фамилию.
Во время Отечественной войны вместе со мной не покидала Ленинград, работала на заводе им.
Карла Маркса диспетчером в цехе, где выпускали детали для  «Катюши».
После войны работала в том же заводе технологом.
Награждена пятью медалями и многими благодарностями за хорошую работу.
До войны прекрасно играла на рояле, с появлением немого кино работала в кинотеатре тапёром.
Хорошо рисовала, обладала чувством юмора, в молодости считалась самой красивой девушкой в Дибунах.
Умерла 14 июня 1994 года.

Дедушка, Трейман Александр-Карл, немец по национальности, настоящая его фамилия была Treuman  (Treu-звучит как «трой» --- верный) родился в лютеранской семье, где было принято наследство передавать только старшему сыну.
Дед был вторым сыном.
Он получил хорошее образование.
После смерти отца старший брат подарил моему деду кусок земли с лесом в Финляндии, и Александр-Карл вынужден был покинуть Германию, и уехать поближе к своей земле.
Он ехал почти налегке, захватив с собой небольшой альбом «Moderne Sichtbilder» c великолепными фотографиями Баден-Бадена 1897 года, которым он очень дорожил.
Может быть, он был оттуда родом?

Дорога к финской границе почему-то проходила через Курскую губернию, где он познакомился с богатым купцом, в доме которого воспитывалась сирота, дочь друга этого купца Афиногенова Анна Алексеевна.
В том же доме жила и родная дочь купца, слабоумная дочь Лушка.
Когда Александр-Карл попросил у купца руки Анны, купец поставил такое условие.
Он согласен на этот брак и отдаст приданое Анны, которое осталось после смерти её родителей, кроме того, прибавит лушкино приданое, но свадьба состоится, только если молодые возьмут к себе Лушку и будут кормить и ухаживать за ней до конца её жизни.
Так эта тихая юродивая и жила вместе с бабушкой и дедушкой и умерла в блокаду от голода.

Раньше граница с Финляндией шла по реке Сестре.
Неподалёку находился посёлок Дибуны, где жили финны, и пять немецких семей.
Русских там тогда не было.
Деду понравилось это место, понравились приветливые сородичи, с которыми можно было разговаривать на родном языке.
(Кстати, немцы называли это место «die Buhne», что означало «сцена»).
Здесь на деньги бабушки был построен дом.
Родились 2 старших сына, погодки, и дочь, моя мама.
Не желая детям такой же участи, как у него, дед крестил своих детей в православной церкви, но учились мальчики в Петри-шуле.

Старший сын, Трейман Евгений Александрович, был матросом на «Авроре».
(Правда,  после того, как был сделан легендарный выстрел).
Он великолепно рисовал, был всегда в хорошем настроении, любил выпить.
У него был сын Юрий и внук Валентин.

Второй сын, Трейман Валентин Александрович, был тихий, ласковый человек.
Во время войны он пошёл добровольцем в ополчение, был ранен и попал в плен.
Он хорошо знал немецкий язык и вероятно сильно разозлил немцев, тем, что воевал против них.
Его нашли в канаве уже мёртвым, все кости были переломаны, а со спины, видимо ещё с живого, была содрана кожа.
У Валентина был сын Виктор, мой ровесник, у Виктора трое сыновей и двое внуков.
Виктор прекрасно рисует и поёт, работал в реставрационной мастерской.

Однажды, ещё до моего рождения, деда арестовали только за национальность.
Описали имущество и собирались сослать семью в Сибирь.
Дядя Женя рассказал об этом своим друзьям, матросам, и они большой толпой пошли требовать справедливости.
Через год деда выпустили, по тем временам это было чудо!
Но вернулся он совсем больной и вскоре умер, не дожив до 60 лет.
Похоронили его в Дибунах.
Я родилась после его смерти.
От мамы слышала, что дед был очень доверчивым, многие этим пользовались, не отдавая долги.
Любил растения и животных, в доме всегда были 2---3 собаки, любил ходить на охоту, но ни разу не смог убить какое-нибудь животное.

Бабушку я помню.
Она всегда чем-то занималась и была удивительно молчаливой.
Умерла она зимой 1941 года и похоронена в братской могиле на Северном кладбище.

{\sloppy

Могила деда находилась на  кладбище ст.
Песочная.
На тёмном камне на немецком языке было написано: «Treuman Alexandr-Karl», дата рождения и смерти.
Мы с мамой бывали там, мама ухаживала за могилой, а я рассматривала красивые готические буквы на соседних камнях.
Сам камень я помню, а вот дату рождения и смерти, к сожалению, нет.
После смерти дяди Жени его жена Вера  похоронила его в том же месте.
Не посоветовавшись с мамой, она выбросила старое надгробие, заменив его на стандартную вертикальную плиту, где русскими буквами были выгравированы  фамилия, изуродованная русской транскрипцией, имя, отчество и даты рождения и смерти дяди Жени.
Имя деда даже не было упомянуто.
Мама была очень обижена на родственников своего брата за неуважение к памяти деда и никогда больше не бывала на этом кладбище.

}

Когда мама умирала, она завещала похоронить себя рядом с папой, что я и сделала.
Прах мамы и папы покоится в колумбарии, секция 13, во внутренней части секции, в верхнем ряду, слева.

Папа, Бочаров Анатолий Васильевич, родился  22 августа 1894 года  в  г. Ростов-на-Дону.
Это данные его паспорта.
На самом деле он родился в Петербурге в 1892 году  и в Ростове-на-Дону никогда не был.
День его рождения может быть, действительно был 22 августа, но не уверена.
Дата смерти --- 1 февраля 1985 года  --- правильная.

Историю папиной семьи  я начну издалека.
После смерти предшественника  Александр III  почти безвыездно жил в Гатчине, его даже прозвали «гатчинский пленник».
Его будущая жена, принцесса Дагмара, в окружении фрейлин прибыла в Петербург.
У них долго не было наследника.
Александр III любил выпить, военные действия его не интересовали, недаром его называли «миротворец», ему нравились молоденькие фрейлины и вскоре одна из австрийских фрейлин забеременела.
Александр срочно выдал её замуж за одного из своих придворных, которому было известно, что невеста ждёт ребёнка.
Родилась девочка.
В семье придворного появилось потом свои дети, и старшая девочка  росла не испытывая доброты и внимания со стороны отчима.
Но зато пользовалась большей свободой и познакомилась со студентом архитектурного факультета Академии художеств, простым мещанином из города Елабуга, Бочаровым Василием Васильевичем.
Молодые люди полюбили друг друга.

Василий приехал в Петербург вместе с братом, который окончил Химико-технологический институт, вернулся в  Елабугу  и работал там главным инженером на сахароварочном заводе.

Когда в семье дочери Александра III узнали, что она, дворянка, хочет выйти замуж за простого мещанина, отчим заявил, что лишает её наследства.
Девушка со слезами прибежала к матери, и та рассказала ей историю её рождения.
Александр III был тогда уже царствующей особой.
Дочь попросила аудиенции.
Царь принял её приветливо, в качестве свадебного подарка подарил ей поместье в Тамбовской губернии, а жениху жаловал дворянский титул и назначил главным архитектором военного ведомства и связи.
Будучи главным архитектором,  Василий Васильевич руководил строительством главного почтамта  в Москве (проект почтамта создавал кто-то другой).
В Петергофе под его руководством строили казармы и церковь.
В закладке первого камня принимали участие Василий Васильевич и царь.

Это событие было сфотографировано.
Фотография долго хранилась в семье, но посторонним людям её не показывали.
К месту закладки камня царственная чета подъехала в карете.
Лошадь оставила  кучу.
Когда фотограф запечатлевал важное событие, на первом плане были царь и главный архитектор, вдали царица и приглашённые.
Случилось так, что на фотографии царица стоит точно над этой кучей.
До революции хранение этой фотографии было небезопасным, т. к. могло рассматриваться как компромат на царицу.
После революции такая фотография, уже по другой причине, тоже могла навлечь беду.
Уничтожена она была после революции.

Семья Бочаровых жила в Петербурге по адресу Чкаловский проспект, дом 34.
Они занимали 1 или 2 этажа (это были либо 2, либо 2 и 3), и в Петергофе был каменный дом, построенный по проекту главы семейства.
Детей было 7 или 8.
Старшие были мальчики, последняя девочка Ольга, была бездетная, умерла в блокаду в Ленинграде.
Мой отец, Бочаров Анатолий Васильевич был младшим сыном.
Одного из братьев звали Леонид,  ещё одного, кажется, звали Павел, имён других братьев я не знаю.
Семья была обеспеченной, подаренное поместье тоже давало доход.
Управляющий регулярно посылал в Питер обозы с овсом, продажа которого шла «на  широкую ногу».
Дети учились в Пажеском  и Кадетском корпусах, у каждого был свой велосипед, в Петергофе рядом с домом  было построено специальное помещение для хранения,  ремонта и ухода за велосипедами.
По тем временам это было роскошью.
Очень красивыми были специальные костюмы для велопрогулок.

Отношения  между родителями и детьми были очень тёплыми.
Иногда приезжали в гости бабушки.
Бывшая фрейлина до конца жизни плохо говорила по-русски.
Она  появлялась всегда в шикарной карете, полной подарков для детей.
Если что-нибудь в поведении внуков ей не нравилось, она забирала подарки и, сердитая, уезжала.
Дети считали её взбалмошной, и однажды, когда бабушка раздала всем подарки, весело переглянулись и вернули их ей, чем опять же рассердили её.
Муж бабушки никогда не появлялся в доме.

Иногда приезжала из Елабуги другая бабушка.
Она собирала всех внуков в гостиной и играла им на рояли классические мелодии.
Играла очень хорошо, но и очень долго.
За окном светило солнце, одиноко стояли велосипеды, а она всё играла, играла.
Папа говорил, что она была строгая, умная и добрая, но уж слишком  увлекалась приобщением внуков к музыке.

Деду, Василию Васильевичу было где-то около 50 лет, когда он заболел инфлюэнцей (кажется, так теперь называется грипп) и умер.

Бабушка очень переживала и, на нервной почве, потеряла зрение.
Но, вероятно как компенсация этого недуга, у неё обострился слух.
Года через 1,5---2 зрение вернулось, но слух до конца её жизни был очень острым.
Пережила деда она меньше чем на 4 года.

{\sloppy

После смерти родителей старшие дети либо учились, либо занимались проблемами организации собственных семей.
Прислуга стала плохо выполнять свои обязанности, воровать.
Мой папа, младший мальчик в семье, взял на себя управление домашним хозяйством.
Некоторое время удавалось ещё кое-как поддерживать порядок в доме, но семья разваливалась.

}

Папа закончил кадетский корпус в Полтаве, получив чин поручика.
Начинал он учёбу в Петербурге то ли в Пажеском, то ли в Кадетском корпусе.
Предполагаю, что он что-то натворил, за что был выслан в Полтаву, но папа никогда не рассказывал причину этой высылки.
Вернувшись в Петербург, он встретил своего друга Сергея Натуса, который закончил Петербургский Кадетский корпус.
Дворянская семья Натусов очень тепло отнеслась к папе.
Вместе они вступили в пожарную группу.
(Тогда это было почётным).
Вместе ездили в Оллело, где молодёжь организовала театральную труппу.
Папа хорошо пел.
В опере «Евгений Онегин» он пел партию Ленского.
В этой труппе  он познакомился с Ольгой Казико, будущей известной драматической московской актрисой.
Младшей в этой труппе была Рина Зелёная.

Папа и Ольга полюбили друг друга.
Была помолвка.
Отец Ольги Казико был хозяином сапожной мастерской.
Старшие братья папы были возмущены желанием потомственного дворянина жениться на дочери сапожника.
Помолвка была расторгнута.

В семье Натусов тоже были неприятности.
Сестра Сергея, Евгения была влюблена в простого ветеринара.
Натусы не знали, что предпринять.
Однажды Сергей пригласил к себе папу, там же была Евгения.
Как бы невзначай их оставили вдвоём.
Через некоторое время Сергей весело позвал родителей и объявил, что Анатолий просил руки Евгении! По тем временам сказать, что это враньё, и Евгения мне не нужна, было бы оскорблением для всей семьи Натусов.
Папа промолчал.
Пришли родители с иконой, благословили молодых.
Потом родилась Тамара, которая была на 25 лет старше меня.

А в это время в России произошла революция.
В Петрограде начались беспорядки.
На улице среди бела дня человек с ружьём мог застрелить другого только за то, что у того, другого, была хорошая шапка.
Со словами: «контра!» он забирал себе эту шапку и шёл дальше под молчание окружающих.

Однажды папа увидел бумажку, приклеенную на стене, в которой было написано: «кто хочет помочь навести порядок в Петербурге, обратиться по адресу ...», дальше был адрес.
Папа пошёл туда.
Встретили его радостно, особенно когда узнали, что он имеет военный чин.
Дали в подчинение группу вооружённых матросов, и в первый же день они ликвидировали какую-то коллективную драку.
Там он стал постоянно работать, познакомился с новыми людьми, с некоторыми подружился.
Ближайшим другом стал Анатолий Васильевич Луначарский.
Папа отзывался о нём, как об интересном, умном, интеллигентном человеке.
Он же дал ему рекомендацию в партию.
Место, где работал папа, называлось ЧК (чрезвычайная комиссия).
Папе довелось бывать в доме Распутина.
Секретарь Распутина был организатором многих беспорядков.
За ним следили.
Папа рассказывал, что в доме висел большой портрет Распутина, и когда арестовывали секретаря, дочери упали на колени перед этим портретом и просили защиты.
Портрет не помог, и секретаря увели.

Папа был хорошо знаком и с Калининым.
Он отзывался о нём как о простецком, добродушном, недалёком человеке.
В те времена часто устраивали «маёвки» т. е. коллективные  пьянки.
Папа и Калинин всегда садились рядом и помогали друг другу незаметно от других выливать водку, чтобы не пьянеть.
Секретарём у папы была племянница Калинина.

Когда папа работал в ЧК, был так называемый дворянский бунт.
В течение одного дня всех дворян решили уничтожить.
Папа предупредил Сергея Натуса, чтобы тот  ушёл из дома, чтобы переждать это время.
Сергей послушался его, переоделся в гражданскую одежду, но забыл снять шпоры с сапог.
Его схватили вечером на улице.
Дворян решили топить в Финском заливе.
На платформы, предназначенные для перевозки песка, сгоняли арестованных.
По краям платформы стояли люди с винтовками.
Платформу отвозили в сторону Кронштадта, открывали и тех, кто пытался выплыть, расстреливали.
Таких платформ было тринадцать.
Отвозили платформы по 2 шт.
Сергей Натус должен был  быть погруженным на 12-ую платформу.
На его глазах погибло много, скорее всего ни в чём неповинных людей.
Он помолился и пошёл на платформу.
В это время из Кронштадта сообщили, что привезли уголь и некому разгружать.
12-ую и 13-ую платформы не стали открывать, а довезли до Кронштадта и заставили грузить уголь.
Какое-то время они у революционных матросов выполняли самую тяжёлую работу, потом забыли, что этих людей собирались ликвидировать, и потихоньку они вернулись в Петроград.

После ликвидации ЧК папу назначили комиссаром по продовольствию, по странному совпадению, в Тамбовскую губернию.
В Петрограде был голод.
Папа решил помочь семье Натусов.
Он вызвал к себе своего бывшего друга, а теперь шурина, и назначил его своим адъютантом.
Голод был позади  Папа с женой и дочерью и Натусы вернулись в Петроград.
Папу назначили первым начальником автогужевого транспорта Петербурга.
Квартира на Чкаловском проспекте подлежала уплотнению.
Папа уговорил Сергея Натуса с семьёй переехать в эту квартиру.

В это время четверо старших братьев папы решили эмигрировать во Францию.
С тяжёлым сердцем они покидали родину.
Большевики вызывали у них ненависть.
Папа тоже.
Перед отъездом они пришли к папе и жестоко избили его, уничтожили все документы.
Три месяца он пролежал в госпитале.
Вернувшись домой, не нашёл жены и дочери, документов и партбилета тоже не было...
Тогда за утерю партбилета могли расстрелять.
Надеяться на влиятельных друзей не приходилось.
Цурюпы и Луначарского, которые давали папе рекомендацию, уже не было в живых, а начальника автогужевого транспорта мог узнать любой кучер.
Папа решил спрятаться где-нибудь в пригороде Петрограда.
Как-то в поезде он познакомился с немцем, который рассказал, что в посёлке Дибуны продаётся маленький домик, что рядом живут только финны и немцы.
Папа подумал, что это подходящее для него место.
К тому же он хорошо знал немецкий, французский, итальянский языки и легко мог общаться с соседями.

{\sloppy

Папа дёшево купил хибарку.
У него были золотые руки, и вскоре она превратилась в славный, уютный домик.
Папа настойчиво искал жену и дочку и наконец нашёл.
Евгения, пока папа был в госпитале, встретила свою старую любовь и вышла замуж.
Теперь её пассия был уже начальником МВД Москвы.
Он удочерил Тамару и дал ей свою фамилию и отчество.
Теперь её звали Тамара Петровна Голубенко.
Пётр Голубенко обещал папе помочь с документами и легко согласился, чтобы Тамара пожила с отцом.
В маленький домик в Дибунах  вместе с Тамарой приехала баба-няня, которая раньше растила Евгению.
Папа зарабатывал разовыми «халтурками», денег вполне хватало, баба-няня очень вкусно готовила, и в домике стали появляться гости.
Первым был Александр-Карл Трейман, потом папа сдружился с его сыновьями и с первого взгляда полюбил дочь Александра-Карла Людмилу, мою маму.
Больше трёх лет он ухаживал за ней.
Тамара подрастала, у Евгении родилась вторая дочь, Регина, и было решено, что Тамара и баба-няня вернутся в Москву.
Уезжая, Тамара посоветовала папе «жениться на тёте Люсе».

}

Он сделал предложение, но мама долго не хотела его принять, может быть, пугала разница в возрасте (12 лет).
Но отец и братья уговорили её, и мама  согласилась.
В то время мама работала секретарём в сельсовете, и брак был оформлен в 1935 году, несмотря на отсутствие у папы документов.
Когда в 1938 г.
родилась я, документов у папы ещё не было, и меня записали на мамину фамилию.

Документы появились уже после моего рождения.
Пётр Голубенко выполнил обещание.
Вроде бы были сохранены имя, отчество и фамилия, но год и место рождения не соответствовали действительности.
Дата дня рождения 22 августа возможно тоже верная.
Папа с явной неохотой  отвечал на мои вопросы, касающиеся его прошлого.
Он привык бояться.
После получения документов мама добавила к своей фамилии папину фамилию и стала Бочарова-Трейман.
Когда я получала паспорт, работница ЗАГСа удивилась, что фамилия отца Бочаров, матери ---
Бочарова-Трейман, а моя только Трейман.
Потребовала принести документы о браке родителей и выписала новое свидетельство о рождении, где моя фамилия была Бочарова.
Потом я вышла замуж за Петрунина Михаила Михаиловича, студента Химико-технологического института и стала Петруниной.
У нас родилось двое детей, Елена и Антон.

Папа с мамой жили очень дружно.
Несмотря на трудную послевоенную жизнь в семье никогда не было скандалов.
Я ни разу не слышала не только грубых слов, но даже никогда никто не повышал голоса.
Иногда папа ездил в Москву, жил в семье Евгении.
Тамара окончила институт иностранных языков, специализация --- венгерский язык; во время войны была переводчицей, вышла замуж за генерала Зайцева  и родила сына Игоря, после войны продолжала заниматься переводами и работать администратором гостиницы «Будапешт».
Тамара и Игорь приезжали в Ленинград, Игорь иногда подолгу гостил у нас.
Папа, когда был в Москве, заходил к Рине Зелёной.
Приезжала к нам и вторая дочь Евгении Регина.
У меня, Тамары и Регины были добрые отношения с детьми Сергея Натуса Михаилом и Мариной.

Марина до конца своих дней жила в одной из комнат коммунальной квартиры, в которую превратилась бывшая папина квартира.
Она писала хорошие стихи, много читала, почти до восьмидесяти лет плавала и играла в волейбол, обладала чувством юмора.

От Миши я многое узнала о папе.
Сейчас ему больше 80 лет, но он сохраняет ясность ума (что в наше время крайне редко).
Он много интересного рассказывал о своих родителях, но это уже не моя родословная.
Обладает чувством юмора и умеет расшевелить и заставить смеяться людей разного возраста и разного культурного уровня.
Женат на Галине, умной, доброй, спокойной женщине.

{\sloppy 
Петрунин Михаил Михаилович, отец моих детей  Елены и Антона, родился 9 февраля 1938 года в г.
Ленинграде.
Окончил Химико-технологический институт и работал в НИИ Нефтехим.
Был красивым, спортивным и умным, но несколько ленивым мужчиной.
Последнее его качество оказалось одной из причин раскола нашей семьи, и я ушла к более активному, но менее умному, красивому  и спортивному Хоткевичу Анатолию Викторовичу.
(О чём потом сожалела.)
После развода с Хоткевичем у нас снова возобновились добрые отношения с Мишей.
У него уже была гражданская жена и ребёнок Алексей.
Миша снова хотел вернуться ко мне, но я слукавила, сказав, что ребёнок ни в чём не виноват, и он должен вырастить ребёнка.
На самом деле я привыкла жить одна и не хотела терять свою свободу.
Умер он в 59 лет тоже от инсульта, как и его отец.
Похоронен рядом с отцом на Северном кладбище.

}

Петрунин Михаил Тимофеевич, отец моего мужа, родился в Башкирии.
Две семьи, Петрунины и Варфоломеевы, русские, неизвестно откуда и по какой причине приехали в Башкирию, построили себе дома и нарожали детей.
Дети вырастали,  женились на представителях другой семьи, потом образовалась целая деревня, которую назвали «Рассвет».
У жителей этой деревни были только две фамилии: Петрунины и Варфоломеевы, все были родственниками.
Мать Михаила Тимофеевича звали Елена Дормидонтовна.
Муж её погиб в гражданскую войну, но почти все её дети получили высшее образование.
Только один сын стал трактористом (уважаемая профессия!) и остался в деревне.
Остальные уехали в Ленинград и Москву, но часто приезжали в «Рассвет».
Умерла она в возрасте 90 лет.
Рассказывали, что в какой-то праздник, который отмечали в её доме, она приготовила всё для гостей, выпила с ними рюмочку, сказала, что пошла помирать и легла на сундук, скрестив руки.
Через несколько минут крикнула гостям: «дайте же мне свечку!»
Все думали, что это шутка.
Дали ей в руки свечку, зажгли её и пошли праздновать дальше.
Она тихо лежала на сундуке, и кто-то из гостей, проходя мимо заметил, что она действительно умерла.

Михаил Тимофеевич окончил сельскохозяйственный институт и вернулся в деревню.
Ещё в институте он познакомился с Николаевой Зинаидой Алексеевной, которая была старше его на 9 лет.
От этого знакомства появился на свет мой муж Миша.
Когда ему было всего полгода, Зинаида Алексеевна привезла его в деревню «Рассвет», пришла в дом Елены Дормидонтовны, посадила малыша на стол, сказала: «это ваш внук!» и уехала в Ленинград.
Потом началась война и голод в Ленинграде.
Она снова приехала в «Рассвет».
Мише было 3 года и он уже ловко пас гусей.
Деревенские относились к ней с осуждением.
Когда война кончилась, она взяла сына и уехала в Ленинград.
Отец поскучал без сына и тоже поехал за ними.
Он работал  председателем  колхоза «Красный партизан», потом председателем совхоза «Пригородный», потом на руководящей работе сначала в Исполкоме, потом  в Смольном.
Какое-то время был главой в Кингисепе.
И снова вернулся в «Пригородный».
У Михаила Тимофеевича была врождённая интеллигентность.
Он вёл себя всегда естественно, был доброжелательным, приветливым.
Люди его любили, он их тоже, особенно женщин.
Умер он в возрасте 52 лет от инсульта и похоронен на Северном кладбище, на участке, принадлежавшем совхозу «Пригородный».

Казалось, полной противоположностью ему была Зинаида Алексеевна.
Она родилась в г. Ливны, в семье приказчика.
Вероятно от отца она унаследовала высокомерное отношение с окружающими и раболепное с начальством.
Всегда  важная, чванливая, она пыталась строить из себя светскую даму.
Читала книгу о правилах хорошего тона, но там, к сожалению, не было написано, что нельзя чавкать.
А как она чавкала!
Громко, с присвистом и причмокиванием!
Слова произносила, как ей казалось, должны были бы говорить дамы высшего света, например, «бэсцвэтный, пионэр» и т. д.
Любила употреблять в речи, часто не к месту, редкие слова, такие как «кощунственный» и др.
Гости бывали у них очень редко.
Друзей у неё не было.
Очень редко приходили сёстры, внешне очень похожие и почти все старые девы.
(Их было 7.)
Только у одной, Пелагеи, была дочь без отца.
Меня свекровь всегда старалась обидеть, унизить, многое делала для того, чтобы мы с Мишей поссорились, но наша любовь всё выдерживала.
Конечно, я её не любила.
Единственное, что я могу сказать о ней с уважением, это её манера готовить еду.
Всегда красиво, безукоризненно опрятно и вкусно, но меню её было всегда одно и то же: на первое --- щи, на второе --- тушёная картошка с отварным мясом из щей и обязательно компот.
Последние дни она плохо понимала, что происходит, плохо ориентировалась в пространстве.
Умерла она, когда ей не было 70 лет.

{\sloppy

Хоткевич Анатолий Викторович к нашей родословной не имеет отношения, но возможно он оказал какое-то влияние на становление характеров  Лены и Антона.
Родился 14 мая 1928 года в Полтаве.
Отец был архитектором по строительству мостов и дорог.
Много работал в Кисловодске.
Рассказывал, что род его идёт от гетмана Хоткевича, (в разных источниках эта фамилия упоминается и как Ходкевич, и как Хаткевич), который вёл польское войско на Русь.
В школьных учебниках эта фамилия есть, но обычно никто не помнит, кто вёл поляков, а помнят только Минина и Пожарского --- предводителей русского войска.
Мать была юрисконсультом.
Девичья фамилия Бокий.
Семья Бокий очень известная.
Брат матери, Глеб Бокий был во время революции вторым, после Ленина.
Фильм «достояние республики» начинается с телеграммы, подписанной Глебом Бокий о необходимости беречь культурные ценности страны.
Глеба Бокия расстреляли, когда ему было 27 лет.
Были ещё два брата.
Один был ректором Горного института, другой заведовал кафедрой экономики.
На памятных досках на здании Горного института увековечены их фамилии.

}

Анатолий Викторович учился в школе, когда началась война.
В войну был сыном полка, потом окончил танковое училище в Ульяновске и с лейтенантскими погонами был послан на службу в Калининград.
Быстро продвигался по служебной лестнице, в 40 лет был уже полковником, служил в Копьяре, был с полком на испытаниях первого атомного взрыва над Самарой, работал на Байконуре, в Ленинграде заведовал учебной частью в академии имени
Можайского.
Был очень энергичным человеком, но круг его интересов замыкался армией.
Книги его не интересовали, он читал только газеты.
Когда он стал отчимом Антона, то пытался приобщить его к спорту, устроил в секцию СамБО.
Когда у Антона проявились математические способности, он организовал застекление всех лоджий и окраску полов в математической школе, с условием, чтобы туда приняли Антона.

Мы с ним много путешествовали по стране.
В Ленинграде он чувствовал себя плохо.
Врачи поставили диагноз --- бронхиальная астма и посоветовали сменить климат.
Для того чтобы разрешили обмен на Крым, нам пришлось развестись.
Я обещала ему переехать в Крым, как только выйду на пенсию.
Когда он уехал в Алупку, я, оставшись с мамой и Антоном, почувствовала себя такой счастливой! Я скрыла от него, что как блокадница я вышла на пенсию раньше на 2 года.
Умер он в 72 года, простудившись после дня рождения.
Похоронен в Ялте.
