%\addcontentsline{toc}{chapter}{Рассказы}
\section*{Бычиха}
\addcontentsline{toc}{section}{Бычиха}


После окончания войны стали возвращаться в Ленинград прежние его жители.
Они приезжали в комнаты, которые занимали до отъезда, а люди, испытавшие весь ужас блокады и переселившиеся из разбомблённых домов в эти же, тогда свободные, помещения, должны были убираться в другое место.
Так случилось и в нашей семикомнатной коммунальной квартире.

В сравнительно большой комнате, наиболее удалённой от кухни, жила добрая старая женщина.
Мы звали её тётя Таня.
Ей нравилось, когда мы, дети, приходили к ней и не отпускала нас без подарков.
И у меня, и у моей подружки Тамары долгое время стояли фарфоровые слоники, подаренные ею.
Но вернулись прежние хозяева и добрая тётя Таня исчезла.
Вместо неё въехали в эту комнату семья Бычковых.

Муж во время войны работал шофёром грузовой машины, а жена была поварихой.
Из Германии они привезли полную машину вещей, отобранных у немцев.
Вся наша квартира невзлюбила их, и особенно жену и называла её не иначе как Бычиха.
Я слышала, как она гордо рассказывала на кухне, что заставляла немок бесплатно стирать её одежду и восстанавливать плиссе юбок.

Вскоре в бывшем тётитанином сарае захрюкал поросёнок Борька.
Бычиха носила пойло животным в сказочном, длинном, василькового цвета, пан-бархатном платье.
Наши мамы продали в блокаду всё маломальски ценное, чтобы выжить и к концу войны ходили в ватниках и кирзовых сапогах.
Представляю, как хотелось им после войны красиво одеться и почувствовать себя женщинами, ведь они тогда были ещё не старыми! 
Моей маме не было ещё 40 лет, а тамарина мама была ещё моложе.
Я помню их грустные глаза, когда они смотрели вслед Бычихе с пойлом.

Когда хозяйки пекли блины или оладьи, все, кроме Бычихи, угощали детей.
А ещё Бычиха не разрешала нам, бегать по коридору и требовала, чтобы мы разошлись по комнатам, или шли на улицу.
Конечно мы её не слушались и всегда были готовы устроить ей какую-нибудь шкоду.

А рядом с кухней жила тётя Маруся.
Она работала подсобницей и растила двоих мальчишек, наших друзей.

Когда Бычиха ставила на плиту кастрюльку с водой и клала туда курицу, потом уходила к себе домой, тётя Маруся тоже ставила рядом свою кастрюльку с водой, перекладывала бычихину курицу к себе, а нас просила, чтобы мы сообщили ей, если Бычиха выйдет из комнаты.

С каким удовольствием мы выполняли это поручение!

Стоило только Бычихе открыть дверь, как мы стремглав бежали к тёте Марусе.
Она быстро перекладывала курицу назад.

Бычиха клала специи в свою кастрюльку, солила и снова уходила к себе.
Курица тут же перекочёвывала в соседнюю ёмкость, а мы снова были начеку.
В результате Славик и Коля получали наваристый бульон, а Бычиха ворчала: «Какие в этом эсэсэри курицы плохие, никакого навара.
Вот в Германии курицы, так курицы!»

Может быть благодаря этим бульонам Коля Лавров вырос здоровым, окончил театральный институт и долгое время был на главных ролях в Малом драматическом театре у Льва Додина.

\section*{Довоенное детство}
\addcontentsline{toc}{section}{Довоенное детство}

Родилась я недалеко от Ленинграда в посёлке Дибуны.
В то время там жили финны и 5 немецких семей, в одной из которых я и появилась на свет.
Название посёлка мои родственники произносили «Дибюне» (нем. --- сцена).
Раньше посёлок располагался на небольшом возвышении и действительно напоминал театральную сцену.

Мне было 3 года, когда началась война и мы уехали в Ленинград, но я хорошо помню расположение комнат и мебели, сад и особенно игрушки: необыкновенный, тонкого фарфора, кукольный столовый сервиз, куклу с фарфоровой головкой, закрывающимися голубыми глазами, плюшевого мишку, который был больше меня ростом, и будку с очаровательным щенком.

Мама и папа работали, и моим воспитанием занимались бабушка и нянька.
Няньку я не любила и старалась убежать к бабушке, которая всегда была чем-нибудь занята, и я по всей вероятности ей мешала.

Зато какими радостными были дни, когда мама и папа были дома! 
С комода папа снимал будку со щенком, я кричала ему: «Трезор!», и он выскакивал из будки.
Он был как живой! 
Папа всегда что-нибудь мастерил в саду, а я пыталась ему помогать: приносила гвозди, проверяла качество готовой работы.
А сад у нас был необыкновенный! 
Его создавал мой дед.

Все друзья и знакомые деда, приезжая к нему в гости, привозили какие-нибудь чудесные растения, они знали, что это будет самым лучшим подарком.
В саду росли 4 сорта малины: красная, чёрная, жёлтая и белая.
Было несколько сортов садовой земляники и даже настоящая клубника с удивительно сладкими ягодами сиреневато-розового цвета.
Гордостью деда был китайский лимонник, который прижился в нашем северном климате.

Самого деда я, к сожалению, никогда не видела.
Он умер до моего рождения, после того, как отсидел год в тюрьме только за национальность.
Дома о нём много говорили, о его любви к порядку, о доверчивости, о любви к природе.
Когда я рвала ягоды в саду, то знала, что это память о дедушке.

Когда к нам приходили гости, родители или бабушка показывали им сад, они внимательно рассматривали его, удивляясь, казалось бы таким привычным для меня, растениям.
Потом все шли в дом, где на белоснежной скатерти, на красивых тарелках стояли приготовленные бабушкой угощения, и тут нянька уводила меня в другую комнату.
Там я залезала на сундук, где сидел огромный, мягкий и уютный мишка.
Я обнимала его и шептала на ушко сердитые слова о своей няньке.
Когда из гостиной слышалась песня, это был для меня знак, что теперь меня оттуда никто не выгонит, я бежала в гостиную.
Мама (позднее я узнала, что она работала тапёром в немом кино) или папа садились за рояль.
Иногда папа пел.
Гостям нравилось его пение, ему долго аплодировали.
Думаю, репертуар его был шире, но запомнились только ария Ленского и песня «далеко-далеко, где кочуют туманы...».
Иногда и я пела «выйду, выйду в чисто поле, посмотрю, какая даль...».

Но детское счастье кончилось 22 июня 1941 года.

Папа ушёл на фронт.
Парголовскую немецкую колонию департировали.
О пяти семьях, которые жили в Дибунах, забыли.
Мама боялась департации и, бросив всё, взяв только самое необходимое, документы, меня и бабушку, уехала в Ленинград, и все ужасы блокады мы с мамой испытали сполна.
Бабушка умерла от голода зимой 1941 года.

\section*{Контузия}
\addcontentsline{toc}{section}{Контузия}

День победы был настоящим праздником.
Наша большая коммунальная квартира преобразилась.
Все улыбались, поздравляли друг друга, обнимались и целовались даже бывшие прежде в ссоре.
Мечты о прекрасном будущем у каждого были разные.
Мама мечтала о возвращении отца и все последующие дни только и говорила о нём, вспоминая различные эпизоды довоенной жизни.
В моей памяти всплывало весёлое папино лицо и как будто слышался его приятный, добрый голос.

Каждый день в детском саду я слышала, что чей-то отец вернулся, а моего так долго не было! 
Прошло больше месяца и вдруг воспитательница сказала: «Нина, беги домой, твой папа вернулся!»

Так быстро я никогда не бегала!

По направлению к нашему дому шёл седой, небритый человек в солдатской шинели.
Он совсем не был похож на образ отца, оставшийся в моей памяти, но, не знаю почему, я побежала к нему с криком «папа!».
Человек поднял меня на руки, внимательно посмотрел на меня, прижал к себе и заплакал.
Я тоже прижалась к нему.

Потом мы сидели за столом в нашей маленькой комнате и мама рассказывала, как она обрадовалась, когда папа позвонил ей на работу, как побежала в проходную, чтобы объяснить, где мы живём и отдать ключи, а потом вернулась в цех, позвонила в детский сад и оформила увольнительную.
Она настойчиво спрашивала, как же я узнала папу, но я ничего не могла ответить.
Я смотрела на незнакомого, уже гладко выбритого мужчину и думала, мой ли это отец.
И решила проверить.
«Папа, спой мне!» 
Он печально посмотрел на меня и хриплым голосом ответил: 
«Нет, Нинуша, не могу.
У меня была контузия» ---
«А что такое контузия?

И папа рассказал мне, как однажды, когда они воевали в Польше недалеко от г. Лодзь, их отряд лежал в поле, прячась в траве.
Вдруг он увидел вдали большую ромашку.
Лепестки её на фоне зелени казались ослепительно белыми, а жёлтая серединка, как маленькое солнышко, так и звала к себе.
И папа, очень любивший цветы, пополз к ней.
В это время началась артатака.
Сильный взрыв.
Очнулся он с ромашкой в руке, а на том месте, где он раньше лежал, была глубокая воронка.

Взрывной волной были выбиты зубы, нарушены голосовые связки и он долго ничего не слышал.
В госпитале ему сделали удобные вставные челюсти, с которыми он не расставался уже до конца жизни, слух вернулся, но папе надо было говорить погромче, а голос почти пропал.
Он говорил с хрипотцой, но иногда, когда он был в хорошем настроении и думал, что его никто не слышит, пел.
Но это пение было скорее похоже на громкое мурлыканье.

С тех пор самым приятным подарком для папы всегда был букет крупных ромашек.

\section*{Школа}
\addcontentsline{toc}{section}{Школа}


Осенью 1945 года я должна была пойти в школу.
Но в сентябре я заболела корью, потом болезнь перешла в воспаление лёгких.
Врачи уже не надеялись, что смогу выкарабкаться.
Просыпаясь, я видела над собой заплаканное мамино лицо.
Но чудо случилось, и я стала поправляться! 
Теперь мне кажется, что чудом был папин подарок к Новому 1946 году, я и сейчас иногда вижу его во сне.

Это был игрушечный дом с настоящими открывающимися окнами и дверями.
Заднюю стенку можно было убрать и увидеть квартиру с кухней, туалетом и ванной и тремя комнатами.
Одна была для бабушки, другая для мамы с папой, а третья для девочки.
Маленькие стульчики, столик, кроватки и члены семьи были так аккуратно сделаны папиными руками! 
А в туалете можно было по настоящему спускать воду, дёрнув за тоненькую верёвочку.
Особенно мне нравилась ванна, в которой можно было по-настоящему в воде купать девочку --- крохотного пупсика.

Тогда мы жили в большой коммунальной квартире без ванны и наша комната на троих была всего 8 кв. м.
Папа говорил, что у нас потом тоже будет такая квартира.

В первый раз в первый класс я пошла в конце января 1946 года.
Бледная, худая, остриженная наголо, с большим синим газовым бантом на шее я вызвала смех у девчонок.
На перемене подошла ко мне второгодница Лариса Абрамова и потребовала отдать ей мой синий бант.
Услышав отказ, она, здоровая, старше меня как минимум на год, вцепилась в меня.
Я дралась как могла.
Ей стали помогать девчонки из класса, их было много.
Кое-кто получил от меня царапины и синяки, но больше всех досталось конечно мне.
Домой я вернулась поцарапанная, но с бантом, хоть и разорванным в клочья.

До поступления в школу я дружила только с мальчишками и в играх всегда была лидером.
Тогда я крепко запомнила правило, что драка в истинном понимании этого слова может быть только один-на-один.
Если одного бьют несколько, это не драка, а избиение и участники избиения уважения не достойны.
Я решила, что и те, которые молча наблюдали за дракой, тоже недостойны уважения.
Обычно общительная, в классе я стала молчаливой, всегда сидела одна за партой.
Каждый год это отчуждение усиливалось.
Однажды было собрание, на котором приняли обязательство создать стопроцентный комсомольский класс.
Я молча посмеивалась про себя, как можно выполнить это обязательство, если в классе есть я.
Несмотря на все уговоры, в комсомол я не пошла.

Учителя обычно не интересовались отношениями между ученицами.
Были такие, которые менторским тоном объясняли мне прописные истины, что надо быть доброй и приветливой.
Я молчала и вспоминала свой первый школьный день.
Но были и совсем другие, которые стали для меня не только наставниками, но и настоящими друзьями, с которыми я общалась и после окончания школы.
Их я часто с глубоким уважением и признательностью вспоминаю, но все же школьные годы я не могу назвать, как в известной песне, чудесными.

Пожалуй, это были самые тяжёлые, после блокады, годы моей жизни, в которых тоже не было ничего радостного.

\section*{Один день}
\addcontentsline{toc}{section}{Один день}

{\sloppy

Проснулась я сегодня как всегда рано.
Серенькое утро.
Облака как грязная вата.
Накрапывает дождь, и настроение соответствующее.
Вспомнила, что Путин подписал закон о замене льгот денежными компенсациями, стало ещё гнуснее.
Сволочи, начали бы с себя, больше бы денег сэкономили.
Так нет ведь, опять бедных стариков грабят.
Хорошо ещё, что дети живут далеко от этой обдолбанной страны и сын может мне помогать.
А то бы сдохла на эту пенсию и вошла бы в число тех, про которых говорят, что в этом году у нас на три миллиона стало меньше людей, живущих ниже прожиточного уровня.

}

Пошла на кухню приготовить завтрак.
Включила радио и ждала, пока закипит кофе.
Диктор говорил о заботе нашего правительства о пенсионерах.
Выключила.
Позавтракала и сложила в раковину грязную посуду.
Горячей воды опять нет.
Её не было 25 дней, потом включают часа на 3---4 и снова выключают.
А деньги берут по полной программе.
Интересно, почему об этом не говорят как о заботе о нас?

Можно было бы сказать, что пожилые люди должны больше отдыхать, а не заниматься уборкой и мытьём посуды и у нас всё делается для здоровья людей.

Сделала лёгкую уборку, оделась и пошла в магазин.
Это теперь по инерции я говорю: «в магазин».
На самом деле надо сказать: «в магазины».
В «Сезоне» я покупаю овощи и сыр, на эти продукты цены ниже, чем на базаре и можно выбрать каждую помидоринку, да и никто не обвесит, в «О'кей»е дешевле шоколад, молочные продукты и хлеб, а в «Пятёрочке» я покупаю в основном продукты для моего кота.
Теперь времени у меня больше, и на этом тоже можно экономить.

Когда отберут льготы, придётся оплачивать транспорт.
Попробовала ходить пешком.
Понравилось.
Но это сейчас, когда тепло, приятно.
А потом, когда будет метель, мороз...

Поход в магазины занял часа три.
Пора и пообедать.
По радио в это время передавали концерт по заявкам.
Я часто слушаю эту передачу и замечаю, что всё меньше звучит патриотической музыки.
Иногда ветераны заказывают песни времён Отечественной войны, но это лирические песни типа «Катюши» или «Синий платочек» а, в основном, чувствуется ностальгия по старым, добрым временам.
Чаще всего просят исполнить «Домик окнами в сад».

После обеда села за компьютер.
Я веду там своё домашнее хозяйство и включила музыку, диск с романсами в исполнении Валентины Пономарёвой.
У этой певицы удивительный голос, но исполнение могло бы быть и получше.
Зато сопровождение двух гитар просто великолепно!

Она запела цыганский романс и я вспомнила своё знакомство с цыганами.
Мне было лет шестнадцать, когда на Сенном рынке ко мне подошли цыганки и предложили погадать, а я сказала: 
«Вы бы лучше спели!» 
Они согласились и повели меня за рыночные ларьки.
Сначала неуверенно, но постепенно входя во вкус, они пели мне песни и сначала требовали оплату, но когда деньги кончились, они продолжали петь просто так.
Лучше всех пела красивая, опрятно одетая цыганка по имени Роза.
Она спросила, люблю ли я цыган, я ответила: «Да, особенно песни!» --- «А вышла бы замуж за цыгана?»

--- Конечно, если бы полюбила.

--- Приезжай к нам в Тосно --- пригласила Роза.
Она дала мне адрес и сказала, что вечерами в субботу они всегда дома.

В первую же субботу, захватив бутылку портвейна, я поехала в Тосно на улицу Кирова.
Нашла просторный деревянный дом, где меня приветливо встретила Роза.
Внутри дома мебели почти не было.
Много места занимал большой деревянный стол и рядом деревянные скамьи.
Постепенно собирались гости.
Два брата Розы хорошо играли на гитарах.
Потом пришёл молодой цыган Ромка.
Он жил где-то неподалёку и его попросили сплясать.
Как же Ромка плясал!

Сначала он как бы нехотя пошёл по кругу, но его медлительность не была вялой, походка была упругой и его тело казалось сжатой пружиной, которая вот-вот вырвется, распрямится и даст волю жгучему цыганскому темпераменту.
Я не могла оторвать от него глаз.
Вечером Ромка провожал меня до вокзала.
Я робела, идя рядом с ним, он казался мне волшебником.
Но разговор его был скушным и неинтересным и очарование постепенно угасало.

Все летние субботы я проводила в Тосно.
И каждый раз, когда плясал Ромка, сердце моё билось учащённо.
Он переводил мне цыганские песни, я даже научилась сама понимать цыганские слова, стала петь вместе со всеми.

Осенью Ромка как всегда провожал меня до вокзала и как всегда разговор его был скушным и даже его ласковые слова «золотая», «жемчужная» вызывали раздражение.
Стоя под фонарём, мы ожидали мой поезд.
Под электрическим освещением ромкины волосы казались темнее и блеск их был больше заметен.
Ромка что-то говорил, а я смотрела на его волосы.
Вдруг на шею выползла вошь.
Она медленно проползла по кривой и снова скрылась в волосах.

Больше в Тосно я никогда не ездила.
Но часто, и особенно когда слышу цыганские мелодии, вспоминаю, уже совсем по-доброму Ромку, Розу и их тосненских друзей.
И вот сегодня эти воспоминания размягчили душу.
Когда проходила мимо зеркала, заметила, что улыбаюсь.

{\sloppy

В хорошем настроении и с удовольствием поужинала.
Сложила грязную посуду в раковину.
Горячей воды так и не было.

}

\section*{Выбор профессии}
\addcontentsline{toc}{section}{Выбор профессии}

После окончания школы вплотную встал вопрос: кем быть? 
Это был трудный вопрос.
Мне всё было интересно и почти все институты привлекали меня, но мой папа был уже пенсионером, у мамы был предпенсионный возраст, и сидеть на шее родителей ещё 5---6 лет мне казалось невозможным.
Важнее всего было побыстрее получить профессию.
ПТУ, которое готовило мастеров индивидуального пошива женской верхней одежды, короче портних, показался наиболее подходящим, хоть и всерьёз я задумывалась о профессии вагоновожатой.
Соблазняла возможность закрыться в кабине и петь весь рабочий день.

Я очень любила музыку.
Играла на домре в оркестре народных инструментов, руководитель оркестра давал мне уроки игры на гитаре, я научилась играть на банджо и мандолине.
А ночью мне часто снилось, что свободно играю на пианино и почему-то всегда только классическую музыку.
Я пыталась поступить в кружок, где меня научили бы игре на фортепьяно, но дома обязательно нужен был инструмент, а в восьмиметровую комнату, в которой мы жили, инструмент поставить было невозможно, да и денег у родителей на такую покупку не было.

А пела я самозабвенно.
Однажды заметила, что папа со вниманием прислушивается к моему пению.
Я спросила: «Ну и как мой голос?» 
Он ответил: «Голос, что в попе волос, тонок, да не чист!» 
После этого я стеснялась петь при всех.

Итак, портновское ПТУ.
В моей группе было 13 девушек, каждая из которых была по-своему очень интересным человеком.
После постоянного ощущения внутренней напряжённости в школе и брезгливо-молчаливого отношения к одноклассницам я попала в приветливый, добрый мир.
Наверное, подобное чувство испытывают люди, оказавшиеся на свободе после долгого тюремного заключения.
Все мы легко усваивали портновское ремесло, а после занятий бежали в филармонию, часто ходили в Эрмитаж и Русский музей, на концерты ансамбля «Дружба».
Жизнь заиграла весёлыми красками!

Словом «петеушница» называют теперь раскованных девиц с интеллектом майского жука.
Я не знаю, как сложилась жизнь пяти иногородних девушек, с которыми связь прервалась, когда они после окончания ПТУ вернулись к себе домой, а из ленинградок --- одна стала искусствоведом, другая костюмером на Ленфильме, была даже одна депутатка, и все после ПТУ получили высшее образование.
А по интеллекту они были на порядок выше моих одноклассниц! 
А какие прекрасные были музыкальные вечера! 
Мы собирались у Леры Грудзенко на улице Декабристов, она так чудесно играла на рояле! 
Я не помню, чтобы кто-нибудь в школе так тонко чувствовал музыку.
Через 1,5 года мне вручили диплом с отличием с присвоением самого высокого разряда мастера индивидуального пошива и направление на работу в ателье.
Началась трудовая жизнь.
И тут я случайно увидела объявление о приёме в вечернюю музыкальную школу имени Римского-Корсакова в класс гитары.
Выдержала конкурс и была зачислена в класс ученицы Иванова-Крамского Ядвиги Ричардовны Ковалевской.
Я была счастлива! 
С большим удовольствием я начала заниматься.

Работа в ателье была двухсменная --- неделю в утро, неделю в вечер.
Мне приходилось пропускать учёбу в неделю, когда была вечерняя смена.
Преподаватели музыкальной школы решили мне помочь и дали рекомендацию на фабрику Луначарского.
Меня приняли настройщицей щипковых инструментов.
Не прошло и месяца, как я получила заказное письмо из управления трудовых резервов с требованием отработать 3 года в системе Ленинградодежды, либо оплатить учёбу в ПТУ.
Сумма за обучение была чудовищно высокой.
Я поехала в управление, просила, плакала, но безрезультатно.
С музыкой нужно было расстаться, по крайней мере на 3 года.
Вернулась в ателье с обидой на весь белый свет.

Стала читать юридическую литературу в надежде найти лазейку, чтобы вернуться к музыке.
Своих книг в этой области у меня, разумеется, не было.
Ходила в библиотеку на набережной реки Фонтанки.
Здание библиотеки и внутренняя обстановка умной тишины завораживали.
До слёз захотелось учиться.
Лазейку не нашла, но обратила внимание на то, что имею право продолжить образование по полученной специальности и на работе мне должны идти навстречу, вплоть до предоставления односменной работы.
Значит я могу поступать в институт, но только в текстильный и только на швейно-трикотажный факультет.

Этой возможностью я воспользовалась.
После отработки трёх лет в ателье перешла на работу в конструкторский отдел предприятия военно-промышленного комплекса и получила право перейти со швейно-трикотажного факультета на механический, который и окончила с дипломом инженера-конструктора.
Работу свою я любила.

Но как заноза в сердце осталась нереализованная любовь к музыке.
Жизнь почти прожита, но и сейчас она прекрасна, потому что можно петь! 
Сначала в церковном хоре, потом в фольклорном немецком хоре «Надежда».
Когда я выхожу на сцену с небольшим ансамблем-квартетом, а иногда и с сольными номерами, чувствую себя такой счастливой!

Я понимаю, что это почти сумасшествие, в моём возрасте брать уроки вокала, но какая же радость чувствовать, как от правильного дыхания становится голос сильнее и диапазон его увеличивается! 
И тогда кажется, что может быть не такая уж я и старая и жизнь играет весёлыми красками!

\section*{Надо учить английский!}
\addcontentsline{toc}{section}{Надо учить английский!}

Собираясь в гости к дочери в Америку, я обзванивала разные авиакомпании, старалась выбрать самый дешёвый билет.
Каждый раз дочь браковала мой выбор.
Одна компания казалась ей ненадёжной, полёт другой компании совершался в неудобное время.
В конце концов она сама выбрала SAS (скандинавские авиалинии).

До Сиэтла должна была быть посадка в Копенгагене.
Меня беспокоило, как я смогу разобраться в незнакомом мне аэропорту и найти рейс Копенгаген --- Сиэтл, ведь я не знаю английского! 
Но дочь меня успокоила:

--- Датчане наверняка понимают немецкий язык, а если нет, то иди туда, куда все пойдут!

В Пулково, в зале ожидания рейса на Копенгаген, я познакомилась с женщиной, которая летела в Париж, сосед в самолёте летел в Испанию.
(Хороша бы я была, если бы за кем-нибудь побежала!)

В Копенгагене я без проблем нашла табло с указанием выхода на посадку в Сиэтл, легко нашла нужный мне зал ожидания.
Сразу поняла, что здесь собрались американцы.
Удобно (но некрасиво) одетые, они сидели в развязных позах на стульях и даже на полу (а свободных мест было много!).
Я уточнила: «Сиэтл?» --- «Ес!» и спокойно села читать книжку.

Вдруг по трансляции на английском языке что-то объявили и все поднялись и двинулись к выходу.
Я ничего не поняла, решила, что изменили зал ожидания до Сиэтла и пошла следом за всеми.
Перед окошком выстроилась очередь.
Все доставали паспорта, я тоже.
Всех пропустили, меня --- нет! 
До времени вылета оставалось полчаса.
Служащий в окошке что-то сказал мне по-английски.
Я спросила по-немецки: «Что случилось?» 
Он развёл руками, показывая, что не понимает и указал мне на скамейку около застеклённой двери с надписью «полиция» и жестом объяснил, чтобы я там села.

Я села, но время шло и тревожные мысли лезли мне в голову, может быть что-то не понравилось в моём паспорте, и причём тут полиция? 
До отлёта оставалось 7 минут.
Я не выдержала.
Подошла к окошечку «полиции» и стала требовать позвать «рашен» (русского) или «джомини» (немца).
Служащий, похожий на итальянца, вышел из кабинки, приветливо посмотрел на меня, расставил широко руки и замахал кистями, изображая самолёт, потом указал на часы и сделал жест как бы перечёркивая их.
Я чуть не заревела, решила, что мой самолёт уже улетел.
Ничего не понимая, я ещё настойчивее стала требовать «рашен енд джомини».
Он указал на скамейку и снова сделал жест, предлагая сесть и успокоиться.
А вот этого я не могла и стала злобно прогуливаться взад и вперёд перед окошком полиции, поглядывая на «итальянца».
А он не торопясь взял пластмассовый треугольник, внутри которого была пустота в виде круга, и стал что-то рисовать.
Я видела, что он старательно занимается изобразительным творчеством.
Ему-то что, попробовал бы он побыть в моей шкуре!

Через несколько долгих минут он вышел с листом бумаги, на котором очень аккуратно были нарисованы часы.
Каждая цифра --- шедевр каллиграфии.
Стрелки указывали 7 часов.
Он снова изобразил самолёт и ткнул пальцем в лист.
Я сказала «но!» 
Достала билет, где было указано время вылета 15 часов.
Он перечеркнул время, указанное в билете и написал «19».
Я подумала, что меня почему-то отправят другим рейсом и стала соображать как быть, если меня не встретят.

От грустных мыслей меня отвлёк служащий, которому «итальянец» указывал на меня.
Он подошёл и заговорил ПО-НЕМЕЦКИ! 
Объяснил, что самолёт задерживается и всех выпустили погулять по Копенгагену, а у меня не было шенгенской визы и я не имела права выйти за территорию аэропорта.
Он предложил мне успокоиться, пройти в кафе и выпить кофе.
Я отказалась, но служащий пытался уговорить меня: 
«это бесплатно, засчет компании SAS! 
Но я не смогла бы ничего проглотить, настолько была взвинчена.
Разговаривал «немец» очень доброжелательно, я постепенно успокоилась и поняла, что сама виновата.
Уезжая в Америку, надо было выучить хотя бы элементарные фразы.
Да, надо учить английский!

\section*{Америка из окна кухни}
\addcontentsline{toc}{section}{Америка из окна кухни}

В Америке, в семье дочери, я по собственной инициативе взялась за приготовление пищи.
Почти весь день я проводила на кухне, окно которой выходило на веранду, но был хорошо виден и весь двор.
Около самого дома стоял огромный старый кедр и на участке было много экзотических растений, названий которых я не знала.
Знакомыми были магнолия и аралия с сильными, крупными листьями, по форме похожими на кленовые.
Это было любимое мамино растение.
В большом горшке оно росло у окна в её комнате и после смерти мамы неожиданно засохло, несмотря на то, что я не забывала его поливать.
А здесь прямо в грунте без всякого ухода прекрасно чувствовало себя это зелёное чудо.

Иногда выходил погулять старый ярко-рыжий кот.
Дочка забрала его с собой, когда навсегда уезжала из Питера.
Как же он красиво выглядел под солнечными лучами на фоне зелени!

Перед окном прыгали белки, они были с пушистым, сероватым мехом и более крупные, чем наши, питерские, и хвосты у них были не круглые, как щётка, а плоские, как будто проглаженные утюгом.
Я выходила на веранду и кормила их нечищенным арахисом.
Иногда я рассыпала арахис на полу веранды и смотрела из окна, как они ловко разбираются с орехами и однажды увидела, что этот корм привлёк внимание больших ярко-синих, отливающих перламутром, птиц.
По форме они напоминали соек.
На более тёмной головке был высокий хохолок и над глазами нежно-голубые вертикальные полосочки.

Птицы садились на ветки рядом с верандой, потом одна из них летела на стол и грозно стучала клювом, чтобы спугнуть белок.
Белки очень неохотно отступали подальше от орехов, птица заглатывала один орех, а другой хватала в клюв и улетала.
На её место прилетала другая.
Иногда белкам надоедало ждать, одна из них вставала на задние лапки и быстро-быстро цокала на птиц, явно ругаясь, но решительный бой дать не решалась.

У знакомых моей дочери я спрашивала об этих птицах, но многие их не видели, или видели, но не знали, как они называются.
Говорили, что они водятся только в Америке.
Я вспомнила, что в сказках герои ищут и пытаются поймать синюю птицу, птицу счастья.
А вдруг «синяя птица» это мечта обиженного, униженного человека убежать в далёкую страну, где чистый воздух и ощущение надёжности, защищённости, то есть счастья!

А ещё во дворе под деревянным настилом жила морская свинка.
Каждое утро я насыпала сухой корм в её миску, приносила морковку и звала:
«Свинюша!»
Она бежала ко мне на своих коротеньких ножках и ела прямо из рук.
Потом из окна я видела, как с разных концов двора торопились к свинкиной миске большие буроватые мышки и лопали свинкин корм.
Однажды я посчитала: из миски торчало 7 мышиных хвостов.
На это зрелище приходил смотреть пушистый дымчатый соседский кот.
Он с интересом наблюдал их возню, но никогда не пытался поймать мышку.
Однажды я увидела как он шёл по забору, а внизу, по перекладине, мышка утаскивала куда-то ворованную морковку.
Я думала, что он вот-вот прыгнет на мышку, но нет, этот лентяй разлёгся на заборе и задремал.

Вечерами к дому приходили еноты, они забавно пили воду из корытца стоя вертикально, как столбики и набирали её в ладони верхних лапок, ну прямо как люди! После них оставались мокрые следы на деревянном настиле, которые были похожи на следы ног ребёнка.
Я умилялась, глядя на их мордочки, как бы украшенные чёрными полумасками.

Утром на мой зов «Свинюша!» никто не выбежал.
Я думала, что свинка спит и оставила корм и морковку около маленького хода под настил.
Сама пошла на кухню и поглядывала из окна.
К миске подошёл большой опоссум с голубовато-серым мехом на спине и белоснежным на груди и голове.
Нос у него был нежно-розовый, таким же был коротенький хвост.
Он съел морковку и, не торопясь, ушёл.
На следующее утро я проснулась от душераздирающего кошачьего крика.
Выскочив во двор, увидала енотов, один из которых стаскивал с кедра соседского дымчатого лентяя, которому было уже поздно помогать.

Поняла, почему не вышла морская свинка.
Обошла настил и увидела глубокий подкоп.
Понимаю, что банально, но как же тесно соседствует красота и жестокость!

\section*{Подарок деда Мороза}
\addcontentsline{toc}{section}{Подарок деда Мороза}

Антошке было уже четыре года.
Каждое утро мама провожала его в детский садик неподалёку от дома.
Близился Новый Год, окна магазинов украшали мишурой, гирляндами разноцветных огоньков.
В детском садике детишки разучивали песенку о ёлочке, слушали сказки про деда Мороза и Снегурочку, а вечерами  он засыпал домашних вопросами:

--- А дед Мороз правда есть, а он добрый, а правда, что он всем детям приносит подарки?

--- А что бы ты хотел? --- спросила мама.

--- Конечно ёлку! --- не задумываясь ответил малыш.

В те годы выполнить его желание было очень сложно.
Папа каждый предпраздничный вечер выстаивал в очереди за ёлками и каждый раз возвращался домой с пустыми руками.

Мама провожала Антошку в детский сад и с завистью глядела на соседний дом, из окна которого на четвёртом этаже на верёвке была подвешена красивая, пушистая ёлка.
Ветер раскачивал её из стороны в сторону и она  кокетливо поворачивалась то одним, то другим боком.

Встречая сына из садика мама слышала один и тот же вопрос:

--- Мама, а дед Мороз ещё не принёс мне ёлку?
--- Нет ещё, сынок, --- отвечала мама, --- он сейчас очень занят, смотри, сколько детей, им всем  надо принести подарки.

--- А мне?

--- И тебе тоже, ты только подожди!

Двадцать девятого декабря погода неожиданно испортилась.
Резкий, холодный ветер подхватывал снег, кружил его.
Небо и земля как будто слились,  свет фонарей был едва заметен.
Возвращаясь из детского сада домой, мама крепко держала за руку сына, почти до бровей замотанного шарфом.

И вдруг что-то большое и  тёмное, падающее с неба,  на миг закрыло тусклый  свет, упало на снег, покатилось прямо к мальчику и остановилось  перед ним.
Это была ёлка!
Малыш  стянул шарф  к подбородку и радостно закричал:

--- Мама, дед Мороз кинул мне ёлку!

Он схватил обрывок верёвки, привязанной к стволу, и потащил ёлку домой.
Мама сразу узнала красавицу с четвёртого этажа, но лишить сына встречи с чудом  она не могла.
Отец, вернувшийся после работы опять без ёлки, сразу почувствовал запах хвои.

--- Как тебе удалось достать ёлку? --- спросил он радостно.

--- Это не мама достала! --- закричал Антошка, выбежав в прихожую, --- это сам дед Мороз мне с самолёта кинул! Он же должен был очень многим ребятам что-нибудь подарить, ему было некогда, а завтра он придёт в наш детский сад и я скажу ему «спасибо!»

На следующий день  дед Мороз действительно пришёл в детский сад и всем детишкам подарил пакетики со сладостями.
Антошка поблагодарил его за ёлку и спросил:

--- А почему Вы, когда кинули мне в снег ёлку, не кинули и эти конфеты?

Дед Мороз сначала удивился, а потом ответил:

--- Так они же провалились бы в снег, и ты бы их не нашёл!

Теперь Антошка уже большой, но мне кажется, до сих пор верит в чудо.

\section*{Теперь мне страшно}
\addcontentsline{toc}{section}{Теперь мне страшно}

Когда началась война, мне исполнилось три года.
Конечно, я многое не помню, вернее, я просто не понимала, что происходит.
Но некоторые события оставили след в моей памяти.

Нас, детей, пытались эвакуировать из Ленинграда.
Мама одела мне на шею цепочку с медальоном (в котором были выгравированы моя фамилия, адрес и дата рождения), отвела на станцию.
Там стоял поезд.
Нас погрузили в товарный вагон, двери загородили скамейкой.
Незнакомая женщина предложила нам сесть на пол, и мы куда-то долго ехали.
Вечером поезд остановился и нас повели к длинному сараю.
Крыша этого сарая в середине провалилась и висела, почти касаясь земли, а справа и слева образовалось нечто, похожее на палатки.
Нас разделили на две группы.
Одну повели в правую, другую в левую «палатку».
Под ногами была солома и на неё нас стали укладывать спать.
Все было очень необычно и, вероятно поэтому, меня не испугал и сильный звук взрыва, от которого мы проснулись.
Взрослые кричали, суетились, и мы бежали с ними обратно к вагонам.
Ещё через некоторое время мы снова были в Ленинграде.

Нас ждали родители.
Когда я вышла из вагона, ко мне бросилась мама.
Она схватила меня на руки, плакала, долго целовала меня.

Позднее я узнала, что бомба попала в одну из «палаток».
Все дети, которые там спали, погибли.
Нас, вероятно, спасла от осколков провисшая крыша сарая.
После этого родители, оставшихся в живых детей, отказались снова отправить нас в эвакуацию, и я всю блокаду прожила в Ленинграде.

В дом, в котором мы жили, попала бомба, когда нас там не было.
Свободных комнат в других домах было много, и мы с мамой поселились в малюсенькой комнате, но в ней была настоящая круглая печка, которая оставалась тёплой даже утром!
Вероятно, и этой печке я должна быть благодарна за то, что жива.

Люди замерзали и падали на улице.
Остальные не имели сил, чтобы поднять их, и проходили мимо, экономя собственные силы.

Несмотря на голод и тяжёлое время, я мечтала о кукле.
Однажды мы с мамой шли по проспекту К. Маркса и я увидела на снегу большую куклу.
Я просила маму поднять её, но мама молча потянула меня за собой.
Когда мы возвращались, я снова подошла к этой «кукле».
Но у неё уже не было щёк!

Испорченная кукла меня больше не интересовала, и я пошла с мамой домой.
Мне не было страшно.
Только теперь я понимаю, что это была не кукла, это был мёртвый ребёнок и щёки его отрезали голодные люди.
Теперь мне страшно, мне очень страшно!

Мама очень боялась, что если она погибнет, меня съедят голодные люди.
А если погибну я, то её дальнейшая жизнь представлялась ей бессмысленной.
Она считала, что либо мы обе должны остаться жить, либо обе умереть.
Поэтому она всегда брала меня с собой на дежурство во время воздушных налётов.

Мы ходили с ней по крыше и смотрели сверху на тёмный город.
Когда падала зажигательная бомба, взрослые быстро хватали щипцами и гасили её в песке.
Из песка вырывались огоньки, похожие на бенгальские.
Мы, дети, (я там была не одна) лопатками засыпали их песком.
Я была уверена, что делаю важное дело.

Через много лет я увидела в музее эти щипцы.
Я не поверила своим глазам.
Они показались мне такими маленькими!
В моей памяти они остались огромными, с меня ростом, или даже больше.
Просто это я тогда была слишком маленькой!


\section*{Размышление о внешнем проявлении\\  возраста}
\addcontentsline{toc}{section}{Размышление о...}

Недавно я ехала в метро.
Напротив меня сидела женщина.
Что-то в её лице показалось мне неприятным, неправильным.
Я стала осторожно рассматривать.
Ярко накрашенные губы на бледном, дряблом лице и круглые, без единой мимической морщинки, подкрашенные глаза.
Такие глаза бывают только у детей, но лишённые лукавинки и любопытства, они выглядели неестественно.
Руки с тёмными, вздувшимися венами без обиняков выдавали возраст.
Похоже, что женщина сделала косметическую операцию, подтяжку кожи вокруг глаз.

Вспомнились другие встречи.
Неподалёку от моего дома, в Удельном парке летом пожилые люди приходят на танцы.
Там можно увидеть бабушек с бантиками на голове в яркой, вызывающей одежде.
Эти бантики, так же как и глаза моей спутницы в метро, контрастировали с обликом их хозяек и совсем не делали их моложе.
Напротив, с неумолимой жестокостью подчёркивали их возраст.

Спутница в метро заметила моё внимание к ней и на накрашенных губах появилась радостная, самодовольная улыбка.
Открылась дверь электрички, и рядом с моей визави села пожилая женщина.
Аккуратно, со вкусом одетая, без следов косметики, с умными, много повидавшими глазами, она выглядела куда привлекательнее и, пожалуй, моложе своей соседки, несмотря на то, что фактический возраст был, вероятно, очень близким.
