\section*{Последние годы мамы}
\addcontentsline{toc}{section}{Последние годы мамы}
В 2017 году, на 79-м году жизни, моей маме поставили диагноз --- болезнь Альцгеймера.
Я смог с этим смириться только через полгода.

Нам удалось перевезти маму в Штаты --- огромное спасибо дяде Сэму.
(Несмотря на то что у мамы была зелёная карточка, она нарушила несколько правил, и её не должны были впустить.)
Она жила в Сиэтле у сестры, но периодически приезжала к нам.
С год мама вела себя вполне адекватно.
Видимо, положительно влияли правильно подобранные таблетки.
Характер у неё был невыносим задолго до этого.
Теперь появилась тому причина, и я меньше злился.

Я решил для себя, что для мамы ближайший месяц важней чем оставшаяся жизнь.
Понимая, что лечения нет, старался держать её в хорошей физической форме.
Мы гуляли с ней от трёх километров в день, иногда по десять.
Во время прогулок мама гораздо больше походила на разумного человека. 
Бывало даже, что у нас был почти интересный разговор, обычно о давно минувшем --- школьное время и раньше.
Тем не менее каждую неделю я видел деградацию.

Кстати, однажды во время такой прогулки мама пыталась выяснить как так получилось.
Ведь она была ведущим инженером, вся семья на ней держалась, почему вдруг она очутилась в роли ребёнка.

У мамы атрофировалось умение думать, оставались эмоциональные реакции.
Если мне (и, наверно, кому угодно) что-то сказать, то появляется реакция --- фраза, готовая выскочить наружу; можно её удержать, подумать и сказать что-то другое, но мама этого уже не могла сделать.

Она помнила б\'{о}льшую часть таблицы умножения, но не была в состоянии сложить пару цифр.
Она не могла вспомнить смысл сложения и умножения.
Не могла сказать, сколько дней в году или месяце.
Чтобы назвать число дней недели, ей приходилось перечислить все дни подряд, загибая пальцы,
а потом считать пальцы.
Вот ещё примеры: «Мы шли на север и повернули направо. Куда мы идём?» --- на этот вопрос она совсем не могла ответить (не просто путала восток с западом),
или «Если вчера была среда, какой будет день завтра?» --- «четверг».%
\footnote{Очень возможно, что людей, которые вовсе не думают больше, чем мы себе представляем.
Если есть сомнения, можно сделать сосредоточенное лицо и спросить: «... вчера была среда, значит, завтра --- четверг. Правильно?» --- если ответят «правильно», то далее говорить нет смысла.}
При этом мама могла участвовать в разговоре, у неё даже была какая-то политическая позиция.
Казалась невнимательной, обиженной или странноватой, но таких людей много.
Хорошо читала вслух, но не могла понять, что кончился рассказ и начался другой.

Однажды я её попросил помочь собрать облепиху.
Работа несложная --- надо собрать ягоды со срезанных веток.
У мамы состояние было довольно тяжёлое, но она справилась с заданием. 
Более того, ей понравилось, просила ещё.
Возможно, что если бы она жила в деревне, то её болезнь не была бы столь заметна.
К сожалению, подобных занятий в современном мире почти не осталось.

Проявления раздражения по отношению к себе она всегда объясняла исключительно «нелюбовью».
Очевидно, этим многие грешат.
Этот ход рассуждений стал для меня символом слабоумия, до этого я не обращал на него внимания.
Кстати, такой ход рассуждений старательно навязывается детям --- идея абсолютного зла, например, в мультиках, ровно об этом --- вредят просто потому, что не любят --- вопрос «зачем» даже не рассматривается.
По-моему, чересчур самовлюблённо думать, что кому-то есть дело до тебя (или твоей национальности). 

Для себя я решил, что откажусь от ухода за мамой, когда она перестанет меня узнавать.
Но случилось удивительное.
Однажды утром я отправил маму на прогулку.
Вид у неё был подозрительный, но я решил не обращать внимания.
Она не вернулась к вечеру.
Без знания языка, она умудрилась прийти в полицию и объявить, что я её бью.

Через несколько дней мне позвонил полицейский.
Он нашёл маму на бензоколонке.
(У мамы на руке был браслет с адресом и телефоном.)
Я ему нахамил ни за что и сказал отправить маму в больницу.
По сути, я отказался от матери, и было это самое гнусное, что я сделал за жизнь.
Через десять минут попросил Лушу позвонить и извиниться за папу.

Видимо, её перевели в убежище для битых жён, есть такое место у нас в городе;
она ушла гулять и потерялась.
Её некоторое время держали в местной больнице.
Обвинения с меня быстро сняли. 
Потом мама перешла под покровительство дяди Сэма --- и снова спасибо ему за это.

Это случилось осенью 2020-го.
Через пару месяцев, убирая в её комнате, я обнаружил бумажку, на которой было написано «полиция --- полис»,
то есть она готовилась к побегу.

В 2021 году она перестала разговаривать.
Моё участие в её судьбе ограничивалось посылкой шоколадных конфет от случая к случаю.

После её ухода жить стало легче всей семье, не только мне.
В принципе уход за мамой был проще, чем уход за двухлетним ребёнком.
Однако безумный человек в доме превращает жизнь в ад.
Когда она ушла, я был на пределе.

Мама умерла 8 декабря 2022 года, на 85-м году жизни.
Нам сказали, что незадолго до смерти она разучилась глотать и последние дни провела под капельницей.
19 мая 2023 года её прах рассыпан на тропе Simpson Reed Grove Trail в Jedediah Smith Redwoods State Park,
красивое место.
Вроде об этом она просила мою сестру.
Церемонию устроили сестра с Сашей, из пяти внуков приехали три: Алёша, Мефодий и Луша.

Есть несколько анекдотов, связанных с её болезненным состоянием.
Возможно, я их добавлю позже, когда вся эта история уляжется у меня в голове.
