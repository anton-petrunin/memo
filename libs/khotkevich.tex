\section*{Атомный взрыв над Самарой}
\addcontentsline{toc}{section}{Атомный взрыв над Самарой}

Есть такие гарнизоны, о которых наслышаны не только люди военные.
Их названия, почитай, на слуху у каждого.
Один из них — Тоцкие военные лагеря.
Своё название получили они от селения Тоцкое, расположенного в верховье реки Самары.
Рассказывают, что основано было оно как казацкий форпост вблизи почтового тракта, связывающего центр России с Оренбургом.
Во второй половине прошлого столетия здесь готовилась к Хивинскому походу воинская колонна в составе которой был герой освобождения Болгарии генерал Михаил Дмитриевич Скобелев.

Помнят Тоцкие лагеря и Великую Отечественную войну.
Здесь размещались части польской армии генерала Андерса.
Позже формировался 1-й Чехословацкий корпус, одним из батальонов которого командовал будущий президент страны Людвик Свобода.
Но есть в «биографии» этих лагерей страница особая.
Долгое время о ней нельзя было говорить.

С начала весны 1954 года здесь развернулось интенсивное строительство.
Начали прибывать эшелоны с военнослужащими и техникой.
Позже пришли составы с животными.
В степи, в загонах разместили верблюдов, овец, лошадей, собак.
Переселили несколько сёл.

Сразу же после завтрака, «вооружившись» лопатами, топорами, кирками, роты и батальоны приступали к инженерному оборудованию позиций.
На топографических картах синим и красным цветами были нанесены линии обороны южных к северных.
Каждая линия обороны состояла из нескольких позиций.
Каждая позиция имела две или три траншеи полного профиля, соединённые ходами сообщений.
Оборудовались блиндажи и землянки для размещения личного состава, деревоземляные огневые точки, окопы для танков, орудий, автомашин, бронетранспортёров и другого оружия и боевой техники.
Люди терялись в догадках: когда и какое будет учение?

Только в середине лета было объявлено, что войска будут участвовать в первом в нашей стране учении с реальным применением атомного оружия.

Нам рекомендовали, а значит, приказали до конца учения прекратить переписку.
Приказ о неразглашении содержания учения доводился под личную роспись каждого участника.
Всем выдали противогазы с защитными светофильтрами, индивидуальные противохимические перевязочные пакеты, защитные костюмы, перчатки, сапоги.

14 сентября 1954 года к девяти часам утра приглашённые заняли места на командной вышке.
Среди них были и руководители стран народной демократии, как говорили тогда.
Проследовали доклады о готовности к учению.
В ясном утреннем небе пролетел самолёт-разведчик.
Следом за ним появился самолёт-носитель.

В 10 часов 35 минут наблюдающие за учением через затемнённые стекла увидели рождение нового «солнца» — и через несколько секунд почувствовали резкий удар воздушной волны.
К небу поднимался огромный гриб.

Находившиеся в укрытиях услышали глухой удар, словно, забивая сваю, ударил гигантский молот.
Земля заколебалась, из щелей посыпался песок.
Телефонные провода были порваны взрывом.
Воздух наэлектризован и непроходим для радиоволн.
С этой минуты учение шло без команд, по заранее спланированному сценарию.

Войска наступавших в колоннах двинулись вперёд. Отряды вооружённые инженерной техникой, расчищали завалы и гасили пожары.
По мере приближения к эпицентру всё больше разрушений.

О существовании траншей, проходивших метрах в 500 от эпицентра, напоминала лишь морщина на земле.
Стены траншей и ходов сообщения сошлись, похоронив находящихся там животных.
Блиндажи и землянки разрушены, техника обожжена и опрокинута.
Танк, находящийся в 300 метрах от эпицентра, опрокинут.
Вся площадь у эпицентра радиусом метров 500—600 голая со спёкшейся землёй.
За войсками прямо к эпицентру проследовала колонна легковых автомашин с присутствующими на учениях.
Им показали эпицентр и разрушения, вызванные взрывом.
Пояснения давали учёные во главе с академиком Курчатовым...

Осенью части покинули район учений.
В городке остался один танковый полк, которым командовал полковник Сапего, артиллерийский полк и военный госпиталь.
Для этих частей начались «мирные будни».
Нужно было готовиться к зиме.
Двухэтажные дома узлов связи переоборудовали под казармы, офицерам и сверхсрочнослужащим разрешили размещаться в
сборно-щитовых домиках.
Начали прибывать семьи.
Открылись магазин военторга, школа, домоуправление.
Начал налаживаться быт.
Тогда мы атома ещё мало боялись, не знали его коварства.

По первому снегу началась охота в районе взрыва на зайцев, глухарей, рябчиков.
В магазине военторга появилось лосиное мясо.
Для отстрела лосей выдавались лицензии.
Но казалось, что лоси сами искали помощи от человека.
Они выходили на охотника.
Один слепой лось жил вблизи эпицентра, его подкармливали сеном, корками хлеба, сухарями.

Летом вид эпицентра изменился, в полукилометре начала расти трава только одного вида: очень похожая на молочай.
К середине лета её высота достигла примерно двух метров.
И она продолжала расти, подальше, виднелись ростки других трав.
Только в трёх-четырёх километрах от эпицентра начиналось разнотравье.
Местные жители косили сено в районе взрыва.
Этим сеном кормили коров.
А мы покупали у них молоко.
Другого не было.
Да и сами жители не знали, что этого нельзя делать, так же как и все мы.
На этом молоке выросли наши дети.

Памятником атомному взрыву остался эпицентр с белой проплешиной, известняковым крестом и вышкой, которую несколько раз переносили на место поближе к центру военного лагеря.
Она служила солдатской чайной, и магазином, и складом, и просто трибуной на стадионе.
Со временем исчез и этот памятник.

Полной неожиданностью для всех нас стал приказ о расформировании частей Тоцкого гарнизона в конце лета 1959 года.
Танки и другую боевую технику перегнали на станцию и оставили зимовать под открытым небом, под снегом и дождём.
Солдат и некоторых офицеров направили во вновь формируемые части ракетных войск.

К сожалению, учение тайным оказалось только для советских людей, возможно, поэтому медицина устранилась от изучения влияния на человека малых доз радиации, от наблюдений за возможностью длительного проживания в зонах заражения с малыми уровнями радиации.
За всё время моей службы в Тоцких военных лагерях, а это около пяти лет, никто ни разу не поинтересовался состоянием нашего здоровья.

У нас не было дозиметров.
Контроль облучения не вёлся.
Мы ели мясо животных, поражённых взрывом.
Мы и наши дети пили молоко, полученное от коров, вскормленных поражённой травой.
В столовых и квартирах печи топились дровами, доставленными из района взрыва.

И до сих пор неизвестно: как же повлияла та атомная бомба на здоровье участников учения?

\begin{flushright}
А. ХОТКЕВИЧ, 
инженер-полковник 
в отставке.
\end{flushright}
